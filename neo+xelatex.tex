\documentclass[ngerman]{dtk}

\usepackage{
babel,
caption,
listings,
lmodern,
hyperref,
shortvrb,
todonotes,
xltxtra,
xspace
}

\begin{filecontents}{\jobname.bib}
@Misc{patente,
  author	= {US-Patente},
  note= {79265 und 79868}
}
@Misc{patentdvorak,
  author	= {US-Patent},
  note= {2040248}
}
@Misc{neo2,
  author	= {Neo Projektseite},
  note= {\url{http://www.neo-layout.org}}
}
@Misc{deergo,
  author	= {de-ergo Layout},
  note= {\url{http://forschung.goebel-consult.de/de-ergo}}
}
@Misc{ristome,
  author	= {ristome Layout},
  note= {\url{http://ristome.de}}
}
@Misc{datahand,
  author	= {DataHand},
  note= {\url{http://www.datahand.com}}
}
\end{filecontents}

\newcommand{\taste}[1]{\makebox{\textsf{#1}}}
\newcommand\neoii{Neo\,2\xspace}
\newcommand\neoio{Neo\,1.0\xspace}
\newcommand\neoioi{Neo\,1.1\xspace}
\newcommand\LuaTeX{Lua\TeX}
\newcommand\LuaLaTeX{Lua\LaTeX}
\newcommand{\fixme}[1]{\marginpar{\vspace*{-.5\baselineskip}\footnotesize\color{red}#1}}
\MakeShortVerb\|

\title{Neo \&\ \XeLaTeX\\Ergonomie und Zeichenvielfalt}
\author{Arno Trautmann\and Dennis Heidsiek\and Christian Kluge}

\address{Arno}{Trautmann}{Boxbergring 10\\ 69126 HD-Boxberg\\ \Email{arno.trautmann@gmx.de}}
\address{Dennis}{Heidsiek}{Brucknerstraße 10\\ 27474 Cuxhaven\\ \Email{dennis.heidsiek@googlemail.com}}
\address{Christian}{Kluge}{Dorfstraße 1\\ 12345 Kleinkötersdorf \Email{ckfrakturfreak@web.de}}
\keywords{Ergonomie, xelatex, Neo} % noch welche rein?

\begin{document}
\setmonofont[Scale=0.8]{DejaVu Sans Mono}
\maketitle
\begin{abstract}
Seit der weiten Verbreitung von \XeTeX\ und der fortschreitenden Nutzbarkeit von \LuaTeX\ ist die \TeX-Welt in der Zeit der Unicode-Kodierung und moderner Schrifttechnologien angekommen. Doch das Haupteingabegerät, die Tastatur, ist bei vielen Nutzern größtenteils noch in der Zeit der mechanischen Schreibmaschinen steckengeblieben. In diesem Artikel soll das moderne Tastaturlayout Neo vorgestellt werden, das eine zeitgemäße Arbeit in der Textverarbeitung ermöglicht und vor allem den Umgang mit \LaTeX\ deutlich vereinfachen kann.
\end{abstract}
Seit den Anfängen des Maschinenschreibens hat sich an der Form und Belegung der Tastaturen wenig geändert: Die Form der Schreibmaschinentastatur wurde für Computer übernommen und seitdem beibehalten. Minimale Anpassungen an einzelne Sprachen machten aus dem amerikanischen „\hbox{qwerty}“ (benannt nach der oberen Tastenreihe von links nach rechts gelesen) das deutsche „qwertz“ und das französische „azerty“. Es stellt sich daher die Frage, warum die Tasten genau auf diese Weise angeordnet sind. Und warum sind sie überhaupt schräg versetzt und nicht in Form einer Matrix angeordnet?

\section{Die Vergangenheit}
Die Antwort darauf liegt weit zurück: Im Jahren 1868 hat Christopher Latham Sholes ein Schreibmaschinenmodell mit der bekannten |qwerty|-Anordnung hergestellt.\cite{patente} Die Versetzung der Tasten war aus rein mechanischen Gründen nötig, damit die Typenhebel für alle Buchstaben Platz fanden. Die häufigsten Buchstaben wurden halbkreisförmig angeordnet, die restlichen Buchstaben dazwischen verteilt. Damit sich die Hebel nicht verkanten, musste darauf geachtet werden, dass häufig nacheinander angeschlagene Tasten (wie |qu|) nicht nebeneinander lagen. Das alles ergibt allerdings keine ergonomische Tastatur, die mit dem 10-Finger-System blind, schnell und angenehm zu bedienen ist, denn man muss recht häufig die Finger aus der Grundstellung bewegen, um die Tasten in der oberen oder unteren Reihe zu erreichen – z.\,B. das sehr häufige „e“ auf der oberen Reihe. Jede solche Bewegung bedeutet aber ein Verlust an Geschwindigkeit und mehr Arbeit für die Finger, was zu schnelleren Ermüdungserscheinungen führt und Gelenkschmerzen verursachen kann.

Schon im Jahre 1932 hat August Dvorak die ungeschickte Belegung als solche erkannt und nach längeren Studien eine neue Tastaturbelegung erstellt, 1936 wurde das Patent \cite{patentdvorak} angenommen. Es folgten einige Jahre später weitere moderne Entwicklungen wie de-ergo und RISTOME, die aber bis heute keine große Nutzerbasis haben. \cite{deergo,ristome} Das liegt einerseits am niedrigen Bekanntheitsgrad alternativer Layouts, andererseits am Problem des Umlernens: Selbst wenn man sich vorgenommen hat, eine neue Belegung zu erlernen (erfahrungsgemäß benötigt man ca. 2 Wochen, um wieder einigermaßen flüssig schreiben zu können), ist es immer ein Problem, an anderen Computern zu arbeiten – der Leser kennt dieses Problem vielleicht vom Umstieg von Word auf \LaTeX\ – eine Investition, die sich aber meist doch sehr gelohnt hat.

\subsection{\neoio und 1.1}
Im Jahre 2004 hat Hanno Behrens eine weitere ergonomische Tastatur entwickelt, die er Neo nannte. Etwa ein Jahr später kam mit \neoioi eine Verbesserung und teilweise Erweiterung des Layouts auf, z.\,B. wurde eine zweite \taste{Alt Gr}-Taste auf der linken Hand eingeführt, um den Zugriff auf Sonderzeichen zu vereinfachen – ein großer Schritt in Richtung Ergonomie beim Programmieren, man denke etwa an die schwere Erreichbarkeit des |\|.
\section{Gegenwart}
Seit dieser Zeit wird \neoii entwickelt, dessen Referenz Ende März eingefroren wurde und das somit stabil ist. \cite{neo2} Gegenüber der ersten Verision wurden viele grundlegende Änderungen vorgenommen, wobei die Belegung der Buchstaben zu großen Teilen gleich blieb. Die Abbildung zeigt die aktuelle Belegung.\fixme{oder auch noch nicht \dots{}} Diese weist nach langen Tests im alltäglichen Umgang vieler Entwickler sowie durch umfangreiche Diskussionen eine optimierte Verteilung der Buchstaben auf.
\begin{center}
\includegraphics[width=.8\textwidth]{ebene2}
\captionof{figure}{\neoii – zweite Ebene mit Großbuchstaben}
\end{center}
Hauptänderung in Version 2.0 war jedoch die Einführung von „höheren Ebenen“ der Tastatur,\footnote{Nur in der Software! Neo ist für deutsche Standardtastaturen entwickelt, s.\,u.} insgesamt ist jede Taste sechsfach belegt: Auf Ebene~1 Kleinbuchstaben und Satzzeichen. (\taste{abc.,}) Diese werden durch einfachen Druck auf die Taste erzeugt. Auf der Ebene~2 sind alle Zeichen, die durch Drücken von Shift (oder „Umschalt“) erzeugt werden: Großbuchstaben \taste{A}, wenige Sonderzeichen \taste{§} und Zeichen wie \taste{»« „“}, die für typographisch korrektes Schreiben (statt \taste{"}) verwendet werden. Ebene~3 wird durch \taste{Alt Gr} verwendet, in dieser Ebene sind Sonderzeichen, die zum Programmieren verwendet werden wie |()\/{}_[ ]^!<>|. Um klare Bezeichnungen zu erreichen, wurde \taste{Alt Gr} bei \neoii in \taste{Mod 3} umbenannt.

Diese Funktionalität bieten fast alle Tastaturlayouts in ähnlicher Form. Neo bietet neben dem Modifikator für Ebene~3 noch einen für eine weitere Ebene, konsequent benannt Mod~4. Diese beitet wenige weitere Zeichen in gewohnter Form, vor allem aber als völlig neues Konzept Steuerzeichen, die normalerweise weit außerhalb des Tastenfeldes liegen, wie die Pfeiltasten für Navigation im Text, Enter, Löschen, Tabulator, Entfernen, Pos1, Ende, Einfügen, Bild hoch und Bild runter. Diese sind alle auf der linken Hälfte der Tastatur, sodass mit der rechten Hand \taste{Mod 4} gedrückt und dann bequem am Text gearbeitet werden kann.

Bei einer normalen Tastatur muss die rechte Hand einen Weg von bis zu 5cm zurücklegen, bis die Pfeiltasten erreicht sind; der Weg zu Bild hoch ist meist noch länger und oft muss man auf die Tastatur sehen, um die Taste schnell zu finden. Das alles stört aber den Schreibfluss\footnote{Natürlich sollte man selten Entfernen/Löschen drücken – aber jeder macht Fehler!} und belastet die Arme durch die ständige weite Bewegung.

Auf der rechten Tastaturhälfte findet sich auf der vierten Ebene eine Wiederholung des Ziffernblockes – alle Zahlen und Rechenzeichen sind somit in der Grundhaltung der Hand verfügbar und man muss weder die obere Zahlenreihe noch den „weit entfernten“ Zahlenblock verwenden. Zum Markieren kann wie gewohnt zusätzlich Shift gedrückt werden. Falls man viele Zahlen hintereinander eingeben muss oder sehr viel im Text (oder einem pdf z.\,B.) navigieren muss, ohne zu schreiben, ist die Lock-Funktion nützlich, die analog zu Caps Lock funktioniert: Drücken beider \taste{Mod 4}-Tasten gleichzeitig lässt die Funktion einrasten und man muss nicht ständig den Finger auf der Taste lassen.

\begin{center}
\includegraphics[width=.501\textwidth]{ebene4}%
\includegraphics[width=.501\textwidth]{ebene6}
\captionof{figure}{Ebene 4: Steuerzeichen/Zahlen, Ebene 6: Sonderzeichen}
\end{center}

Damit ist die Vielfalt von Neo noch nicht erschöpft: Für kurze griechische Wörter in einem deutschen Text, vor allem aber für Formelvariablen gibt es die fünfte Ebene, die durch gleichzeitiges Drücken von \taste{Shift}, \taste{Mod 3} und z.\,B. \taste{a} bedient wird\footnote{Man gewöhnt sich schnell an diese Kombinationen, ohne die Finger zu verrenken.} – es resultiert ein \taste{$\alpha$}. Das klingt nicht sehr ergonomisch, ist aber immer noch einfacher, als auf ein ganz anderes Layout umzustellen, nur für ein paar Zeichen; für griechischen Fließtext ist natürlich ein rein griechisches Layout zu bevorzugen. Neben den griechischen Kleinbuchstaben (die mit allen nötigen Akzenten versehen werden können!) sind noch wenige andere Zeichen auf dieser Ebene.

Schließlich bietet die Ebene~6, zu erreichen über gleichzeitiges Drücken von \taste{Mod 3} und \taste{Mod 4}, noch eine weitere Vielfalt an Sonderzeichen, vor allem mathematischen Symbolen und einigen griechischen Großbuchstaben.

Nach dem gleichen Ebenensystem ist der Nummernblock belegt, der außer den Zahlen (erste Ebene) und den Steuerzeichen (vierte Ebene, redundant zu denen auf der vierten Ebene linke Hand) mathematische Relationen und Symbole enthält.

\subsection{Anordnung der Tasten}
Wie kann man sich denn nun all diese Tasten merken? Oft fällt es schon schwer, sich alle normalen Tasten zu merken, aber dann gleich 6 Zeichen pro Taste?

Hier hilft die sinngemäße Anordnung: Die Taste |a| hat die Zeichen |a A { α ⇣ ∀|. Vier davon kann man sich leicht merken. Auf den Formel-Ebenen wurde darauf geachtet, dass die Formelzeichen mnemonisch mit der Buchstabentaste in Bezug stehen.  Die Sonderzeichenebene ordnet die Zeichen nach Sinngruppen; so ist etwa jedes der Zeichenpaare |()<>{}[]›‹‚‘| jeweils mit Zeige- und Mittelfinger einer Hand in der gleichen Tastenreihe einzugeben.  Die Steuerungsebene schließlich enthält Cursor- und Ziffernblock in der von gängigen Tastaturen gewohnten Anordnung an der Grundstellung der beiden Hände. (Mit Erweiterungen: links neben der linken Pfeiltaste ist \taste{Pos 1}, also „ganz links“, oberhalb dieser Taste ist \taste{Bild hoch}, also „ganz hoch“ etc.

Es wird versucht, das Layout so zu gestalten, dass es sehr intuitiv verwendet werden kann und man kein „Raketentechniker“ sein muss, um es zu verstehen.\footnote{Das Keyboarddesign „Space Cadett“ hatte dieses Problem, aber auf Hardwareseite.}

Dank der großen Zeichenvielfalt im Unicode reichen aber die sechs Tastaturebenen nicht aus, um alle mehr oder weniger häufig gebrauchten Zeichen abzudecken. Daher gibt es weiterhin die unter Linux und Solaris bereits übliche Compose-Taste: Drückt man \taste{Mod 3} und \taste{Tab}, danach (zwei oder drei) weitere Tasten, so werden diese sinngemäß „kombiniert“ und ergeben ein neues Zeichen. Die Eingabefolge \taste{Compose a e} ergibt das Zeichen \taste{æ}, die Folge \taste{Compose :)} ergibt das Unicode-Zeichen für ein Smiley. Das klingt wiederum sehr kompliziert und unergonomisch, doch wenn man das Zeichen oft benötigt, lohnt es sich, einmal in der Liste nachzusehen, wie es erzeugt wird und es dann stets direkt eingeben zu können, statt in einem Formeleditor o.\,ä. nachzuschlagen.

\subsection{Neo ist keine Tastatur \dots{}}
\dots{}\,sondern eine Tastaturbelegung. Neo orientiert sich an den Tasten einer handelsüblichen deutschen Tastatur, wie in den Abbildungen angedeutet. Man muss lediglich einen Treiber installieren, der auf der Homepage verfügbar ist und was meist nur ein oder zwei Klicks erfordert. Zur Zeit gibt es Treiber für Windows, Linux und Mac OS X.\footnote{Wegen Softwareproblemen sind in Mac OS X momentan die 5. und 6. Ebene über andere Tastenkombinationen und die 4. gar nicht verfügbar.}

Optimale Ergebnisse erhält man natürlich durch die Kombination einer ergonomischen Belegung und einer ergonomisch geformten Tastatur, z.\,B. Matrixtastaturen (PLUM) oder Konzepten wie DataHand. \cite{datahand}\fixme{Referenz für PLUM}

\section{Neo, Unicode und \XeLaTeX}
Der \TeX-Nutzer wird sich nun fragen, was das mit ihm zu tun hat: Die ganzen Zeichen kann man auch als \TeX-Befehle über ASCII eingeben, man braucht weder Unicode noch \XeTeX\ und vor allem kein Neo. Das stimmt auch prinzipiell. Aber damit wird der Quellcode ziemlich unleserlich. Wer ein |\"a| schöner findet als ein |ä| im Quellcode, muss diesen Artikel eigentlich nicht weiterlesen. Deutlich wird die bessere Lesbarkeit vor allem im Formelsatz, der meist nur schlecht auf einen Blick zu erfassen ist. Man vergleiche die beiden folgenden Eingaben:
\begin{verbatim}
  a) \[\int_{-\infty}^\infty d\Omega \left|\sqrt{\frac{3}{8\pi}}
                          \sin\theta e^{i\phi}\right|^2 \geq 0\]

  b) \[∫_{-∞}^∞ dΩ \left|√{\frac{3}{8π}} \sin ϑ e^{iφ}\right|² ≥ 0\]
\end{verbatim}
% Für die Ausgabe der Formel mit dem angegeben Code – wir wollen ja nicht schummeln! … Außer ein wenig kerning-Korrektur
\def\square{^2}
\catcode`\∫=\active
\catcode`\Ω=\active
\catcode`\√=\active
\catcode`\ϑ=\active
\catcode`\φ=\active
\catcode`\π=\active
\catcode`\²=\active
\catcode`\≥=\active
\let∫\int
\letΩ\Omega
\let√\sqrt
\letϑ\theta
\letφ\phi
\letπ\pi
\let²\square
\let≥\geq
\def\d{\mathrm{d}}
Mit wenigen kurzen Definitionen erhält man jeweils das gleiche Resultat:
\[∫_{-∞}^∞\kern-.6em \d Ω\ \left\|√{\frac{3}{8π}} \sin ϑ e^{iφ}\right\|² ≥ 0\]
\catcode`\∫=12\catcode`\≥=12 %für die Verwendung in \verb
Zur Zeit muss man noch manuell definieren, dass |∫| wie ein |\int| behandelt wird und |≥| wie |\geq|, aber das (noch) experimentelle Paket \Package{unicode-math} von Will Robertson wird hier Abhilfe schaffen und den direkten Umgang mit unicode-kodierten otf-Matheschriften ermöglichen.

Auch für typographisch korrekte Zeichen beim normalen Schreiben bietet Neo – in Kombination mit einem unicodefähigen \TeX-System wie \XeLaTeX, \LuaLaTeX\ oder \ConTeXt\ – die Vorteile der direkten Eingabe. Man muss nicht mehr |--| schreiben, um den Halbgeviertstrich oder Gedankenstrich „–“ zu erhalten, sondern kann ihn einfach eingeben (Ebene~2), ebenso der Geviertstrich oder englische Gedankenstrich (Ebene~3) „—“ statt |---|. Auch Anführungszeichen, die man mit \Package{babel} |"` "'| schreiben muss (was der Autor sich noch nie merken konnte \dots{}), bietet Neo als Zeichen, sowie englische Anführungszeichen und Guillemets: „ “ “ ”  » «. Die Auslassungspunkte, typographisch korrekt mit |\dots| statt |...| geschrieben, sind ebenso vorhanden: …

Definiert man sich den Aufzählungspunkt |•| oder die |¹| als den Befehl |\item|, kann man die üblichen \LaTeX-Umgebungen etwas „ungewohnt“, aber vielleicht übersichtlicher und einfacher schreiben:

\begin{minipage}{.4\textwidth}
\begin{verbatim}
\begin{itemize}
  • erster Punkt
  •[2.] zweiter Punkt
  •[⇒] dritter Punkt
\end{itemize} 
\end{verbatim}
\end{minipage}
\hfill
\begin{minipage}{.4\textwidth}
\begin{verbatim}
\begin{enumerate}
¹ erster Punkt
¹ zweiter Punkt
\end{enumerate} 
\end{verbatim}
\end{minipage}

Das optionale Argument wird dabei wie gewohnt behandelt. Mit etwas mehr Aufwand kann man sogar erreichen, dass nicht mal mehr |\begin{itemize}| und |\end{itemize}| geschrieben werden muss. Für die nächste Ausgabe der DTK ist ein Artikel dazu geplant.

\subsection{Besonderheiten}
Zum Abschluss seien ein paar Spezialitäten erwähnt, die den \TeX-Nutzer vielleicht im Besonderen freuen dürften:

In der DTK 3/2008 hat Markus Kohm berichtet, dass das große scharfe S normiert wurde – schon lange vor dieser Meldung stand bei Neo fest, dass der Buchstabe aufgenommen wird – leider gibt es (noch) sehr wenige Schriften, die das Zeichen besitzen, z.\,B. die Linux Libertine. Aber man kann es mit Neo direkt eingeben.

Liebhaber der gebrochenen Schriften kommen auch auf ihre Kosten, da das lange s (|ſ)| ebenfalls in Neo vorhanden ist; mit einer richtig kodierten gebrochenen Schrift kann man also einfach „deutſch“ schreiben – natürlich auch in Antiqua.

Dank der vielen „toten Tasten“, also diakritischen Zeichen, sind alle Buchstaben aller europäischen, lateinisch basierten Alphabete sowie Griechisch schreibbar – optimal für vielsprachigen Satz.
\fixme{Noch weiter spezielle Anmerkungen?}

\section{Ausblick}
Nicht nur durch die Ergonomie, die jedes Schreiben am Rechner angenehmer macht, sondern auch durch die große Anzahl an leicht erreichbaren Sonderzeichen kann \neoii die Arbeit mit modernen \TeX-Varianten also sehr viel einfacher machen und erleichtert die Lesbarkeit fremden Codes. Damit ist endlich die Zeit von active characters für Standardeingaben vorbei, und die \TeX-Welt ist in der Zeit moderner Kodierungen angekommen.

Doch damit ist die Entwicklung von Neo noch nicht abgeschlossen. Ähnlich wie das \LaTeX3-Projekt gab es schon während der Endphase von \neoii Bestrebungen, die Konzepte nochmals von Grund auf zu überarbeiten und eine Belegung zu entwickeln, die sich auch nach ergonomischer Hardware richtet und daher mehr ein „ganzheitliches“ Konzept darstellt. Diese Entwicklung wird allerdings noch einige Jahre in Anspruch nehmen und soll niemanden davon abhalten, \neoii zu erlernen.

\bibliography{\jobname}
\end{document}